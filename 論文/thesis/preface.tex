%まえがき

%1節
\section{研究背景}
近年,航空機や自動運転などにのシステムの機能安全を保障するために,
半導体デバイスの運用時に故障の有無を検査するフィールドテストが求められる.
パワーオンセルフテスト(POST)は、一つの代表的なフィールドテスト技術である.
POSTはシステムの起動時にデバイスに対して検査を行い,
短時間(システムの起動時間,およそ数十ms)により多くの故障を検出する必要がある.
そのため,より検出精度の高い手法を求め研究が行われている.

故障検出手法の1つにマルチサイクルテスト\cite{multicycle}がある.
マルチサイクルテストは,キャプチャ動作時に複数回のキャプチャサイクルを与えることで,
各サイクルで得られた値を次のキャプチャサイクルのテストパターンとして再利用する手法であり,
従来のテスト手法と比較して,より多くの故障検出の機会が与えられるため,POSTの性能向上に有効な手法である.

%2節
\section{研究目的・目標}
本研究の目的は,システムの機能安全保障のために,
マルチサイクルテストを用いたPOSTにおける遅延故障に対する検出能力を向上することである.

先行研究では,マルチサイクルテストにおける縮退故障の検出モデルを解析し,
縮退故障の検出能力を向上するために回路内に故障を容易に検出できるようなテスト容易化技術を提案した.
しかしながら,遅延故障は時間に関わる故障であり,故障の検出方式は縮退故障と異なるため,
マルチサイクルテストによる遅延故障に対する検出効果が明確ではない.
そこで,本研究の目標は,マルチサイクルテストにおける遅延故障のテスト方法について検討し,
その検出効果を評価する.さらに,マルチサイクルテストにおける遅延故障の検出能力を向上するために,
一部のFFにキャプチャ後の値を強制的に反転させるFF制御回路を挿入するFF-CPI技術を導入し,
その効果を評価する.

%3節
\section{本論文の構成}
本論文は以下のような構成となっている.
第1章では,本研究に至った背景や研究目的,目標について述べる.
第2章では,本論文を閲覧するにあたり必要となる用語について,
第3章では,マルチサイクルテストの概要及びマルチサイクルテストの利点と欠点について述べる.
第4章では,FF制御について述べる.
第5章では,本研究における実験内容及び実験結果について述べ,
第6章では,実験に対する評価,考察を述べる.
また,第7章では,本研究を通してのまとめを記載している.
